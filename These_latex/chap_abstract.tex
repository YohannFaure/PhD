{ \let\cleardoublepage\relax
\chapter*{Abstract}}
\thispagestyle{empty}

When two tectonic plates in contact move relative to each other, an increasing shear force is applied to their interface, the seismic fault, which eventually starts slipping, giving rise to an earthquake that releases the stored elastic energy. The study of these movements on a global scale is carried out in situ using seismic or geodetic measurements, but as well as being highly heterogeneous, the interfaces studied are not directly accessible. Laboratory experiments can be used to study the behaviour of model faults formed by samples of rocks or plastic materials by isolating the parameters that play a role in their dynamics. The aim of our study is to identify the influence of a layer of gouge, a crushed rock present at the interface in powder or grain form, on the behaviour of a seismic fault.


To do this, we developed an experimental device for shearing a frictional interface, enabling us to measure deformations and track particles at high frequency. This allowed us to perform various experiments on two-dimensional frictional interfaces formed by solid blocks of PMMA in the presence of a granular medium trapped at the interface, modelling the gouge. In particular, we studied the interaction mechanisms between a slowly slipping portion of a fault and the surrounding locked regions. Our results show that the slowly slipping portion acts as a precursor to rupture events, destabilising the interface by a mechanism similar to that of a pre-crack in a homogeneous solid. This result and this versatile experimental set-up pave the way for new perspectives in the study of complex frictional interfaces.

