\phantomsection
\chapter*{Table des symboles}
\addcontentsline{toc}{chapter}{Table des symboles}
\markboth{TABLE DES SYMBOLES}{TABLE DES SYMBOLES}

\vspace{-1.5cm}

\begin{tabularx}{\textwidth}{lX}
\textbf{Symboles mathématiques} \\
\\
$\mathbf{G} = \mathcal{F}(g)$ & Transformée de Fourier de $g$\\
$G^*$ & Complexe conjugué de $G$\\
$G\circ H$ & Produit de Hadamard $G$ et $G$\\
$[x]$ & Tenseur dont les valeurs sont les $x_{ij}$\\
$\operatorname{Tr}([x])$ & Trace du tenseur $[x]$\\
$\mathbf{I}_3$ & Matrice identité d'ordre 3\\
$\delta_{ij}$ & Symbole de Kronecker\\
$\Theta(x^n),\,\mathcal{O}(x^n),\,o(x^n)$ & Notations de Landau\\
$\mathscr{C}$ & Chemin dans l'espace\\
$W_\mathscr{C}$ & Travail le long du chemin $\mathscr{C}$\\
$L_\mathscr{C}$ & Longueur du chemin $\mathscr{C}$\\
%
%
\\
\textbf{Mécanique générale }\\
\\
${F_N}$ & Force normale\\
${F_S}$ & Force cisaillante\\
${F_f}$ & Force de frottement\\
${F_r}$ & Force de réaction normale\\
$\mu_s$ & Coefficient de frottement statique\\
$\mu_d$ & Coefficient de frottement dynamique\\
$\mu$ & Coefficient de frottement \\
$A$ & Aire de contact macroscopique\\
$A_r$ & Aire réelle de contact\\
$A_\mu$ & Aide de contact d'un microcontact\\
$f_n$ & Force normale sur un microcontact\\
$f_s$ & Force cisaillante sur un microcontact\\
$[sigma]$ & Tenseur des contraintes\\
$[\varepsilon]$ & Tenseur des déformations\\
$\sigma_Y$ & Contrainte limite d'élasticité (\textit{yield stress})\\
$\sigma_r$ & Contrainte résiduelle -- résistance au cisaillement\\
$g$ & Accélération normale de la pesanteur terrestre\\
$k$ & Constante de raideur d'un ressort\\
$\omega_0$ & Pulsation\\
%
%
\\
\textbf{Propriétés des matériaux }\\
\\
$E$ & Module d'Young\\
$\nu$ & Coefficient de Poisson\\
$E^*={E}/{1-\nu^2}$ & Module d'Young réduit\\
$G_s={E}/{2(1+\nu)}$ & Module de cisaillement\\
$c_{r,\,s,\,p}$ & Vitesse des ondes de Rayleigh, cisaillement et compression\\
$\lambda$ & Conductivité thermique\\
$K_c$ & Ténacité\\
%
%
\\
\textbf{Modèle Rate-and-State }\\
\\
$\theta$ & Âge des contacts\\
$\theta_0$ & Constante de normalisation de l'âge des contacts\\
$\tau=\theta/\theta_0$ & Temps de contact normalisé\\
$V$ & Vitesse de glissement\\
$V_0$ & Constante de normalisation de la vitesse de glissement\\
$D_c$ & Distance caractéristique de renouvellement des contacts\\
$A\text{ et }B$ & Paramètres du modèle Rate-and-State\\
%
%
\\
\textbf{Mécanique de la fracture }\\
\\
$W$ & Énergie élastique totale d'un système\\
$U$ & Énergie nécessaire à la propagation d'une fracture\\
$G=-\partial W/\partial\ell$ & Taux de restitution d'énergie\\
$\Gamma$ & Énergie de fracture\\
$e_{el}$ & Densité d'énergie élastique\\
$f_{ij}(\theta)$ & Fonctions angulaires\\
$\Sigma_{ij}^\textsc{m}(\theta,v)$ & Fonctions angulaires dynamiques\\
$K$ & Facteur d'intensité des contraintes\\
$\ell$ & Longueur d'une fracture\\
%
%
\\
\textbf{Sismologie }\\
\\
$M_L$ & Magnitude locale\\
$M_S$ & Magnitude d'ondes de surfaces\\
$m_b$ & Magnitude d'ondes de volume\\
$M_w$ & Magnitude de moment\\
$M_0$ & Moment sismique\\
%
%
\\
\textbf{Dispositif expérimental }\\
\\
$GF$ & Facteur de jauge\\
$C_\sigma$ & Contraste de chargement\\
$\ell_{eye}$ & Longeur de l'œil granulaire\\
$\ell_{patch}$ & Longueur du patch glissant\\
$t_k$ & Instant d'un évènement de glissement rapide\\
$t_k^+ = t_k+\tau^+$ & Temps limite d'un évènement de glissement rapide\\
$t_k^- = t_k-\tau^-$ & \\
$\delta_{tot}$ & Glissement total à l'interface\\
$\delta_{IE}$ & Glissement inter-évènement cumulé\\
$S$ & Glissement inter-évènement cumulé normalisé\\
$S_f$ & Glissement inter-évènement cumulé normalisé convergé\\
$d_{nuc}$ & Distance de nucléation\\
$x_{nuc}$ & Position du point de nucléation\\
$\phi$ & Couplage d'une interface\\
\end{tabularx}





%$R_r$ et $U_r$ & résistance et tension d'une jauge de déformation au repos\\
%$\vec{X}$ & vecteur de norme $X$\\
%$\norm{\vec{X}}$ & norme du vecteur $\vec{X}$\\
%$v$ & vitesse\\
%$\vec{u_v}$ & vecteur unitaire de la direction de la vitesse\\
%$m$ & masse\\
%$T_{ss}$ & période d'un mouvement de stick-slip\\
%$x,\,y,\,z$ & coordonnées dans un repère carthésien\\
%$\vec{u_x},\,\vec{u_y},\,\vec{u_z}$ & vecteurs de base d'un repère carthésien\\
%$\vec{u_1},\,\vec{u_2},\,\vec{u_3}$ & vecteur de base d'un repère carthésien différent\\
%$\rho_r$ & rayon de courbure\\
%$[\sigma]$ & tenseur des contraintes\\
%$\sigma$ & contrainte le long d'un axe précisé dans le texte\\
%$[\varepsilon]$ & tenseur des déformation\\
%$\varepsilon$ & déformation le long d'un axe précisé dans le texte\\
%$\theta,\,\alpha\,\beta\,\gamma$ & contexte géométrique : angles\\
%$X_c$ & valeur critique d'une grandeur $X$\\
%$l$ & longueur\\
%$l_0$ longueur à vide d'une ressort\\
%$\varepsilon_\perp$ & déformations le long de l'axe d'application d'une force\\
%$\varepsilon_\parallel$ & déformations perpendiculaire à l'axe d'application d'une force\\
%$\sigma_\perp$ & contraintes le long de l'axe d'application d'une force\\
%$\sigma_\parallel$ & contraintes perpendiculaire à l'axe d'application d'une force\\
%$[\sigma]$ & tenseur des contraintes\\
%$[\varepsilon]$ & tenseur des déformations\\
%$\sigma_{ij}$ & composante du tenseur $\sigma$ selon la direction $ij$\\
%$\mathcal{V}$ & volume\\
%$x_c$ & valeur critique de la quantité $x$\\
%$g_{ij}^n$ & fonctions angulaires générales du développement de Williams\\
%$x^\textsc{m}$ ou $x_\textsc{m}$ avec $\textsc{m}=\textsc{i,\,ii,\,iii}$ & valeur de $x$ en mode de fracture \textsc{m}\\
%$\alpha_i$ & $\sqrt{1-(v/c_i)^2}$\\
%$K^S$ & facteur d'intensité des contraintes statique\\
%$K(v) = K^Sk^d(v)$ & facteur d'intensité des contraintes dynamiques\\
%$A(v)$ & facteur dynamique du taux de restitution d'énergie\\
%$D$ & amplitude maximale d'un sismogramme\\
%$D_0$ & amplitude de référence\\
%$N_{>M}$ & distribution frquence-magnitude cumulée\\
%$I$ et $U$ & courant et tension\\










