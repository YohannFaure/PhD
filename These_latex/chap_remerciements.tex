{ \let\cleardoublepage\relax

\chapter*{Remerciements} }
\thispagestyle{empty}


Voici venu l'exercice difficile des remerciements. On va pas se mentir, remercier toutes les personnes qui m'ont accompagné, parfois aidé, beaucoup soutenu, et souvent supporté pendant ces dernières années serait long. Si vous n'êtes pas dans la liste, vous pouvez m'envoyer un message pour m'engueuler, on en profitera pour se planifier une sortie en ville !


\vskip 2cm

\subsection*{La Recherche}
Tout d'abord, chers membres du Jury, J.\,Crassous, J.\,Scheibert, M.\,Métois et F.\,Passelègue, je vous remercie d'avoir accepté d'évaluer mon travail. Vous faites partie des rares personnes qui ont lu ce manuscript en entier, et vos remarques ont été précieuses pour sa finalisation. En m'ayant jugé digne du titre de Docteur, vous m'avez offert l'honneur de me considérer comme un collègue, ce qui, au delà des formalités de mise dans ce type d'exercice, me touche sincèrement.

\vskip 1cm

Elsa,
\vskip .5cm

Voilà maintenant trois ans que nous naviguons ensemble. Trois années durant lesquelles tu m'as ouvert la porte du monde de la recherche et m'a guidé dans ses eaux tumultueuses. Tu savais dès le départ que je comptais après ma thèse quitter cet océan pour retourner patauger dans la mer \texttt{tranquille} de l'Éducation Nationale, et pourtant tu as tout fait pour me pousser à donner le meilleur de moi-même, m'encourageant à y rester. Nous avons bien sûr eu nos différents, et avons dû nous ajuster à nos méthodes respectives pour conserver le navire à flôt, mais le mât à tenu bon. Ces trois années passées au large, nous avons été secoués par de nombreuses lames de fond, les difficultés expérimentales chroniques (le trigger...), des bizarreries informatiques (dont je reste persuadé dans la grande mauvaise foi qui me caractérise que la moitiée au moins est imputable à MatLab), et des contre-temps récurrents (non je n'ai toujours pas lu ce papier que tu m'as envoyé le 29  mars 2021, il est dans la to-do list). Les courants et alizés nous ont également portés dans la bonne direction, lors de l'arrivée de la Peter-box, puis de la JP-box, aux premiers résultats, puis vers la rédaction de l'article et bien-sûr jusqu'à la fin de l'aventure. La manip que nous avons construit est un magnifique bateau de Thésée, j'espère qu'il te servira bien. Prend en soin, je m'y suis attaché à force !


\vskip 1cm

Au laboratoire j'ai fait la rencontre de personnes merveilleuses, que j'aimerais ici remercier. Bien sur les gestionnaires du labo, en particulier Erika et Nadine, ont un rôle tout particulier dans cette thèse pour tout le travail qu'elles font, mais surtout pour tout ce qu'elle font alors que ce n'est pas leur travail. Il a été très agréable de travailler avec l'équipe d'Ingénierie Mécanique, en particulier avec Marc. Avoir accès à un service composé de personnes aussi compétentes et humainement appréciables est une chance incroyable. Pascal et Julian, de l'équipe d'Ingénierie Électronique, ne sont bien sûr pas en reste ! J'ai passé plus de temps dans votre atelier que dans mon bureau, et j'y ai trouvé des collègues dévoués, amicaux, et d'une compétence à l'épreuve de mes demandes les plus saugrenues. Cette thèse n'aurait pas été possible sans le concours de ces trois équipes de choc ! Merci aussi à Caroline et Audrey, qui m'ont tiré vers le haut quand j'étais au plus bas. Vous avez probablement sauvé ma thèse quelque part autour de Septembre 2023. Merci enfin à tous les collègues avec qui j'ai échangé tout au long de cette thèse, Jean-Christophe, Mokhtar, Sylvain, Valérie, et tous les autres.

Je garde bien sûr une petite place très spéciale dans mon cœur (et dans mon estomac) pour la team 11h45, Youssef, Lucien, Rémi et Aubin, déjà cités (plusieurs fois), mais également Marc, grand fondateur, Archimage et Cybermonarque de la confrérie, Geoffroy, personne la plus douce et fondamentalement gentille que j'ai rencontré, Alex, Baptiste, Coralie, Garance, Geoffrey, Julien, Marie, Marlysa et tous les autres.



\vskip 1cm

\subsection*{Les Amitiés}

Il a des amitiés qui s'effacent lentement, d'autres qui reviennent périodiquement à chaque anniversaire ou rentrée scolaire, ou encore des amitiés passionnées qui brûlent d'un feu ardent mais se consumment vite. Mais il y a également des amitiés qui restent, des personnes sur qui l'on pourra toujours compter peu importe la distance, qui forgent qui nous sommes et nous accompagnent à tout moment, les bons comme les mauvais.

Mathilde, nous avons souffert la prépa ensemble, éveillés jusqu'à pas d'heure dans des délires fiévreux déclenchés par un DM trop dur, éveillant les Grands Anciens par des rituels payens. C'était fun.
Les Bitch, Aurore et Fanny, même si on ne se croise plus depuis un bon moment maintenant, vous restez dans mon cœur.
Solène et Joël, meilleure paire de parrain-maraine imaginable, gauchiasses de premier plan, pour vous \textit{amour} rime avec \textit{toujours}.
Aubin, tu m'as apporté au cours de cette thèse un soutient moral indéfectible et a été une source majeure de distraction. J'y serai pas arrivé sans toi.
Youssef, Lucien et Henry, je vous met dans le même panier parce que vous formez un trio démoniaque dont la puissance cumulée dépasse les 9\,000, et à tous les 4 je suis sûr qu'on pourrait lancer une sitcom, mais même individuellement vous êtes des dangers pour la sécurité nationale. Continuons à parler de problèmes de société comme des personnes civilisées (c'est à dire de droite).
Élodie, dont les longues tirades et les soirées gossip autour d'une Délired resteront longtemps dans ma mémoire.
Flora, Vinciane, Sabine et Gilles, vous formez à vous quatre un petit groupe avec lequel les discussions sont toujours passionnantes, variées, drôles, et toujours trop courtes.
Rémi, K.
Paul, tu es un humain très choupi, et tu es toujours le bienvenu chez moi à l'occasion d'un de tes rares passages en France. Puisses-tu continuer à envoyer des cartes postales aussi longtemps que serons loin l'un de l'autre. À vous toustes je dédie ces mots :

\begin{flushright}
\textit{Qui vient en amis arrive toujours trop tard et part toujours trop tôt !}
\end{flushright}



\vskip 1cm


Mon aventure à l'ENS de Lyon n'a (heureusement) pas été qu'académique. Grâce à un certain Jacquelin j'ai essayé un saxophone pour la première fois en 2017. Depuis ce jour, je n'ai pas passé une semaine sans souffler dans un tuyau ! À ce titre je remercie la Fanfarovis, fanfare \textit{officielle} de l'ENS de Lyon, et tous ses membres. En particulier j'aimerais remercier les chefferies successives d'avoir mené ce groupe à ce qu'il est (et supporté mes ingérences). On ne peut pas résumer la fanfare à sa chefferie, merci donc à
Anna,
Achille,
Alban,
Alexi,
Antonin,
Aubin,
Claire (aka Chiot Labrador Sous Cocaïne),
Clara,
Élodie,
Enzo,
Flora,
Geoffrey,
Héloïse,
Isaac,
Julien,
Léontine,
Louis,
Manu (T où ?),
Marie,
Marius,
Octave,
Pauline,
Philémon,
Provisoire,
Rémi,
Robin,
Ronan,
Simon,
Sophie,
Thomas,
Thibault,
l'autre Thibault,
et tous les autres que j'ai pu oublier. J'ai aussi était intégré dans les cercles ludiques, et assisté à la genèse du BuL. Merci donc à ENSecte et tous ses membres, en particulier
AAA,
Antoine,
Aubin,
Eden,
Guillaume,
Henry,
Joël,
Lison,
Lucien,
Nicolas,
Youssef,
et tous les autres.


\vskip 2cm

Enfin il me faut remercier tous ces amis avec qui j'ai tant de bons souvenirs,
Anna,
Aurelia,
Aurélien,
Danaé,
Diane,
Hadrien,
Killian,
Marthe,
Mathilde,
Nacim,
Vianney,
et tous les autres. À toi qui lit ces mots et n'est pas cité\ptmed e, si tu as lu jusqu'ici c'est probablement que je compte pour toi, et que tu comptes pour moi. Merci.




\vskip 2cm



\subsection*{La Famille}




\vskip 2cm


Lucia,
\vskip .5cm

Voilà maintenant 11 ans que j'attend de pouvoir te retourner tes remerciements, c'est dire à quel point te rencontrer à influencé mes choix de vie. Tu m'as mis en tête l'idée saugrenue que les études supérieures c'était bien, et qui l'eut cru, tu avais raison ! Je me souviens encore des longs moments où l'on jouait à Pokémon sur le canapé dans le salon, et où tu me montrais fièrement tes progrès. Je comprend maintenant que toi aussi tu regardais les miens. Alors t'en penses quoi, Docteur c'est pas mal non ?

\vskip 1cm

Valéria,
\vskip .5cm

Dire que tu m'as encouragé à bosser serait une litote, étant donné que tu m'as quand même fait courber l'espace-temps (et surtout la carte scolaire) en m'envoyant au lycée Montaigne, où tu faisais ta khâgne. C'est sans doutes le meilleur choix d'orientation que j'ai fait, puisqu'il a entrainé mon entrée en prépa, puis à l'ENS de Lyon, puis jusqu'ici. \textit{Je voudrais qu'on fût soigneux de choisir un conducteur qui eût plutôt la tête bien faite que bien pleine} comme dirait l'autre, avec toi y'avait les deux ! Tu as également été mon premier contact avec la photographie (ta passion pour les argentiques m'avait fasciné à l'époque), les langues (je dois avouer que mon Espagnol rouille chaque jour un peu plus), les grands auteurs, la sociologie et la politique, qui me passionnent encore aujourd'hui !

\vskip 1cm

À toutes les deux, merci. Si j'avais eu des grandes sœurs j'aurais aimé que ce soient vous.

\vskip 2cm


Maman,
\vskip .5cm
Ce serait trop long de te remercier pour chaque chose que tu as fait, et puis s'épancher sur nos sentiments c'est pas trop dans la famille, alors je vais juste t'écrire un \textit{merci} dont je t'assure qu'il a un poids tout particulier. J'espère que tu es fière de moi, parce que moi je suis fier d'être ton fils.

\vskip 2cm




\vskip 2cm

Margot,

\vskip .5cm

\begin{flushright}
\textit{Grou !}
\end{flushright}







%
%\begin{center}
%\textit{
%Partir, c'est mourir un peu,\\
%C'est mourir à ce qu'on aime :\\
%On laisse un peu de soi-même\\
%En toute heure et dans tout lieu.\\
%\,\\
%C'est toujours le deuil d'un vœu,\\
%Le dernier vers d'un poème ;\\
%Partir, c'est mourir un peu.\\
%C'est mourir à ce qu'on aime.\\
%\,\\
%Et l'on part, et c'est un jeu,\\
%Et jusqu'à l'adieu suprême\\
%C'est son âme que l'on sème,\\
%Que l'on sème à chaque adieu...\\
%Partir, c'est mourir un peu.\\
%}
%
%\end{center}
%
%
%\begin{flushright}
%\,\\
%\,\\
%Edmond Haraucourt
%\end{flushright}


