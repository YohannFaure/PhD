{ \let\cleardoublepage\relax
\chapter*{Résumé} }
\thispagestyle{empty}

Lorsque deux plaques tectoniques en contact se déplacent l'une par rapport à l'autre, une force de cisaillement croissante s'applique à leur interface, la faille sismique, qui finit par céder, donnant lieu à un séisme qui libère l'énergie élastique stockée. L'étude de ces mouvements à l'échelle planétaire s'effectue in situ par des mesures sismiques ou géodésiques, mais les interfaces étudiées, en plus d'être hautement hétérogènes, ne sont pas directement accessibles. Les expériences de laboratoire permettent d'étudier le comportement de failles modèles formées par des échantillons de roches ou de matériaux analogues en isolant les paramètres jouant un rôle dans leur dynamique. La problématique de notre étude est d'identifier l'influence d'une couche de gouge, roche broyée présente à l'interface sous forme de poudre ou de grains, sur le comportement d'une faille sismique.

Pour ce faire, nous avons développé un dispositif expérimental de cisaillement d'une interface frictionnelle permettant des mesures de déformations et du suivi de particules à haute fréquence. Cela nous a permis d'étudier différentes interfaces frictionnelles bidimensionnelles, formées par des blocs solides de PMMA en présence d'un milieu granulaire piégé à l'interface, modélisant la gouge. En particulier, nous avons étudié les mécanismes d'interaction entre une portion de faille en glissement lent et les régions avoisinantes bloquées. Nos résultats montrent que la portion en glissement lent agit comme un précurseur aux évènements de rupture, déstabilisant l'interface par un mécanisme similaire à celui d'une pré-fissure dans un solide homogène. Ce résultat, et ce dispositif expérimental polyvalent, ouvrent la voie à de nouvelles perspectives dans l'étude d'interfaces frictionnelles complexes.


